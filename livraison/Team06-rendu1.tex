\documentclass{article}
\usepackage{luatextra}
\usepackage{polyglossia}
\usepackage{ulem}
\usepackage{framed}
\usepackage{color}
\usepackage{geometry}
\usepackage{amsmath}
\usepackage{unicode-math}
\usepackage[hidelinks]{hyperref}
\usepackage{latexsym}
\usepackage{pdflscape}
\usepackage{pdfpages}
\usepackage{enumitem}
\usepackage{titlesec}
\usepackage{lastpage}
\usepackage{fancyhdr}

\usepackage{ifluatex}
\ifluatex
  \usepackage{pdftexcmds}
  \makeatletter
  \let\pdfstrcmp\pdf@strcmp
  \let\pdffilemoddate\pdf@filemoddate
  \makeatother
\fi
\usepackage{svg}


\setmainlanguage{french}
\selectlanguage{french}
\setdefaultlanguage{french}
%%\setmainfont{Latin Modern Roman}
\setmainfont{Roboto}

\geometry{margin={1in,1in}}


\setlist{nosep} %% No space between lists' items

\pagestyle{fancy}
\fancyhead[R]{}

%% <current page>/<total pages> footer
\cfoot{\thepage/\pageref{LastPage}}

\newcommand\image[2]{
\directlua{
local image = img.scan({filename = "#1"})

image.height = image.height * #2
image.width  = image.width  * #2

node.write(img.node(image))
}
}


%%%%%%%%%%%%%%%%%%%%%%%%%%%%%%%%%%%%%%%%%%
%% \subsubsubsection command definition %%
%%%%%%%%%%%%%%%%%%%%%%%%%%%%%%%%%%%%%%%%%%


\titleclass{\subsubsubsection}{straight}[\subsection]

\newcounter{subsubsubsection}[subsubsection]
\renewcommand\thesubsubsubsection{\thesubsubsection.\arabic{subsubsubsection}}
\renewcommand\theparagraph{\thesubsubsubsection.\arabic{paragraph}} % optional; useful if paragraphs are to be numbered

\titleformat{\subsubsubsection}
  {\normalfont\normalsize\bfseries}{\thesubsubsubsection}{1em}{}
\titlespacing*{\subsubsubsection}
{0pt}{3.25ex plus 1ex minus .2ex}{1.5ex plus .2ex}

\makeatletter
\renewcommand\paragraph{\@startsection{paragraph}{5}{\z@}%
  {3.25ex \@plus1ex \@minus.2ex}%
  {-1em}%
  {\normalfont\normalsize\bfseries}}
\renewcommand\subparagraph{\@startsection{subparagraph}{6}{\parindent}%
  {3.25ex \@plus1ex \@minus .2ex}%
  {-1em}%
  {\normalfont\normalsize\bfseries}}
\def\toclevel@subsubsubsection{4}
\def\toclevel@paragraph{5}
\def\toclevel@paragraph{6}
\def\l@subsubsubsection{\@dottedtocline{4}{7em}{4em}}
\def\l@paragraph{\@dottedtocline{5}{10em}{5em}}
\def\l@subparagraph{\@dottedtocline{6}{14em}{6em}}
\makeatother

\setcounter{secnumdepth}{4}
\setcounter{tocdepth}{4}

%%%%%%%%%%%%%%%%%%%%%%%%%%%%%%%%%%%%%%%%%%
%% end \subsubsubsection definition     %%
%%%%%%%%%%%%%%%%%%%%%%%%%%%%%%%%%%%%%%%%%%

\title{COO — Équipe 6}
\author{Cancela Joël\\Bounouas Nassim\\Mortara Johann\\Novac Pierre-Emmanuel}


\begin{document}
% a

\maketitle
\tableofcontents


\section{Choix de conception}

\section{Diagramme de cas d'utilisation}

\vspace{-5em}
\hspace*{-8em}\includegraphics[scale=1.5]{use_case}
\vspace*{-4em}

\section{Cas d'utilisation: Enregistrer un emprunt}

\noindent\textbf{Nom:} Enregistrer un emprunt \\
\textbf{Description:} Un individu souhaite emprunter un livre.\\
\textbf{Précondition:} \\
\textbf{Postcondition:} Le livre est emprunté.\\
\textbf{Cas d'erreur:} Le livre n’existe pas, le livre est déjà emprunté, l’individu n’existe pas, l’individu est suspendu ou l’individu a déjà emprunté 3 livres.\\
\textbf{État du système en cas d'erreur:} L’emprunt n'est pas validé.\\
\textbf{Acteurs:} La bibliothécaire \\
\textbf{Déclenchement:} La bibliothécaire reçoit une demande d'emprunt d'un livre de la part d'un individu.\\
\textbf{Scénario primaire:}
\begin{enumerate}
	\item La bibliothécaire entre le numéro du document et le numéro de l'individu dans l'interface de la Bibliothèque.
	\item[1] La Bibliothèque recherche l'individu dans l'Annuaire.
	\item[2] La Bibliothèque recherche le livre dans le Fonds de bibliothèque.
	\item[3] Le livre est disponible, l'étudiant n'est pas suspendu et a moins de 3 emprunts.
	\item[4] La Bibliothèque enregistre l'emprunt.
\end{enumerate}

\noindent\textbf{Scénario alternatif:}
\begin{itemize}
	\item[2'.] L’étudiant n’existe pas, fin du cas d’utilisation.
	\item[3'.] Le livre n’existe pas, fin du cas d’utilisation.
	\item[4'.] Le livre est déjà emprunté, l’individu est suspendu ou l’individu a déjà emprunté 3 livres, fin du cas d’utilisation.
\end{itemize}


\section{Diagramme de classes}

\vspace{-5em}
\hspace*{-9em}\includegraphics[scale=1.5]{class}
\vspace*{-4em}

\section{Diagrammes de séquence}

\subsection{Enregistrer un emprunt}
\vspace{-5em}
\hspace*{-10em}\includegraphics[scale=1.56]{sequence_enregistrer_un_emprunt}
\vspace*{-4em}

\subsection{Enregistrer un retour}
\vspace{-4em}
\hspace*{-9em}\includegraphics[scale=1.5]{sequence_enregistrer_un_retour}
\vspace*{-4em}

\subsection{Rechercher un document}
\vspace{-4em}
\hspace*{-9em}\includegraphics[scale=1.5]{sequence_rechercher_un_document}
\vspace*{-4em}

\subsection{Réserver un livre}
\vspace{-5em}
\hspace*{-10em}\includegraphics[scale=1.56]{sequence_reserver_un_livre}
\vspace*{-4em}

\subsection{Relancer pour rendu du livre}
\vspace{-4em}
\hspace*{-9em}\includegraphics[scale=1.5]{sequence_relancer_pour_rendu_du_livre}
\vspace*{-4em}


\section{Diagrammes d'état}

\section{Auto-évaluation}
\begin{itemize}
	\item \textbf{Cancela Joël 95pts}
		Joël a travaillé sur le diagramme des cas d'utilisation, le diagramme de cas d'utilisation détaillé avec Johann, le(s) diagrammes d'états et la rédaction du rapport.
	\item \textbf{Bounouas Nassim 95pts}
		Nassim a travaillé sur le diagramme de classe, le diagramme de séquence "Enregistrer un retour" et la rédaction du rapport.
	\item \textbf{Mortara Johann 95pts}
		Johann a travaillé sur le diagramme de séquence "Enregistrer un emprunt", le diagramme de cas d'utilisation détaillé "Enregistrer un emprunt", et le diagramme de séquence "Réserver un livre".
	\item \textbf{Novac Pierre-Emmanuel 115pts}
		Pierre-Emmanuel a travaillé sur le diagramme de classe, les diagrammes de séquence "Relancer pour rendu de livre", "Rechercher un document" et sur la rédaction du rapport.
\end{itemize}

\end{document}
